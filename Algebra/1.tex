\documentclass[12pt,a4paper]{article}
\usepackage[truedimen,margin=25truemm]{geometry}
\usepackage{amsmath}
\usepackage[dvipdfmx]{hyperref}
\usepackage{pxjahyper}
\usepackage{amssymb}

\begin{document}
\section*{2章 終結式}
\subsection*{定義2.1}
  \[f(X) = a_0 X^m + a_1 X^{m-1} + \cdots + a_m \ a_0 \neq 0 \\
  g(X) = b_0 X^n + b_1 X^{n-1} + \cdots b_n \ (b_0 \neq 0) \\
    \in  K\lbrack X \rbrack \text{に対して}\]

  \[ \begin{pmatrix}
      a_0 & \cdots & 0 & b_0 & \cdots & 0 \\
      \vdots & \ddots & \vdots &  \vdots & \ddots & \vdots \\
      0 & 0 & a_0 & 0 & 0 & b_0 \\
      a_m & 0 & 0 & b_n & 0 & 0 \\
      0 & \ddots & \vdots & \vdots & \ddots & \vdots \\
      0 & 0 & a_m & 0 & 0 & b_n \\
  \end{pmatrix} \]

  を$f \times g$のシルベスター行列と呼び、その行列式を$f$と$g$の終結式 (resultant)とよび、$R(f,g)$で表す

  \paragraph{証明}
    $f(X)$と$g(X)$が共通因子を持つ$\Leftrightarrow$ \\
  ${}^\exists h(X), t(X) \in K \lbrack X \rbrack  \\
    \deg h < \deg g \\
    \deg t < \deg f \\
  f(X) h(X) = g(X) t(X) \\
  \Leftrightarrow \\
  {}^\exists h(X) = C_0 X^{n-1} + \cdots + C_{n-1} \neq 0 \\
  t(X) = d_0 X^{m-1} + \cdots + d_{m-1} \neq 0 \\
  f(X) h(X) = g(X) t(X) \\
  \Leftrightarrow \\
  {}^\exists (C_0, \ldots , C_{n-1}, d_0, \ldots ,d_{m-1} ) \neq \vec{0}$
  共通因子を$s(X)$とすると \\
  \[ \begin{array}{lc}
    a_0 C_0 = b_0 d_0 & X^{m+n-1} \text{の係数} \\
    a_1 C_0 + a_0 C_1 = b_1 d_0 & X^{m+n-2} \text{の係数} \\
    a_2 C_0 + a_1 C_1 + a_0 C_2 = b_2 d_0 + b_1 d_1 + b_0 d_2 & X^{m+n-3} \text{の係数} \\
    \vdots & \vdots \\
    a_m C_{n-1} + a_{m-1}C_{n-1} = b_n d_{m-2} + b_{n-1} d_{m1} & X \text{の係数}\\
    a_m C_{n-1} = b_n d_{m-1} & X^0 \text{の係数}
\end{array} \]
\subsection*{定理 2.2}
  $R (f,g) \in \langle f(X), g(X) \rangle$ \\
  実は$R (f, g) = h(X) f(X) + t(X) g(X)$
  $h(X), t(X)$の係数は$a_0, \ldots , a_m, b_0 , \ldots , b_n$の整式 (整数係数多項式)
  \paragraph{証明}
    $R(f, g) = 0$なら$h(X) = t(X) = 0$とおけばよい\\
    $R(f, g) \neq 0$とする。 \\
  \[ \begin{pmatrix}
      a_0 & \cdots & 0 & b_0 & \cdots & 0 \\
      \vdots & \ddots & \vdots &  \vdots & \ddots & \vdots \\
      0 & 0 & a_0 & 0 & 0 & b_0 \\
      a_m & 0 & 0 & b_n & 0 & 0 \\
      0 & \ddots & \vdots & \vdots & \ddots & \vdots \\
      0 & 0 & a_m & 0 & 0 & b_n \\
  \end{pmatrix}  
  \begin{pmatrix}
    C_0 \\ \vdots \\ C_{n-1} \\ d_0 \\ \vdots \\ d_{n-1}
  \end{pmatrix} = 
  \begin{pmatrix}
    0 \\ \vdots \\ 0 \\ 1
  \end{pmatrix}\]
  の解$(C_0, \ldots , C_{n-1}, d_0, \ldots , d_{m-1})$に対し、\\
  $h^\prime (X) = C_0 X^{n-1} + \cdots + C_{n-1}, t^\prime (X) = d_0 X^{m-1} + \cdots + d_{m-1}$とおくと \\
  $h^\prime (X) f(X) + t^\prime(X) g(X) = 1$がなりたつことにほかならない \\
  $R(f, g) \neq 0$なので、解はただひとつ存在して、それはClamerの公式によって \\
  $C_i = \frac1{R(f,g)}$\\
  よって、各$C_j, d_j$に$R(f,g)$をかけたものは、$a_0, \ldots , a_m , b_0 \ldots , b_n$の整式になる \\

  よって$ \begin{array}{l}h(X) = h^\prime (X) R(f, g) \\ t(X) = t^\prime (X) R(,f, g) \end{array}$とおけばよい

\subsection*{定理 2.3 (拡張定理)}
  \begin{itemize}
    \item $K$ : 代数的閉体
    \item $ I \subset{} K \lbrack{} X_1, \ldots, X_n \rbrack{} $ : イデアル
  \end{itemize}
  \[ \mathbf{V}_k (I \cap K \lbrack X_1, \ldots, X_{n-1} \rbrack ) \ni ( C_1, \ldots , C_{n-1} )  \\
  t_0 (C_1, \ldots , C_{n-1} )\neq 0 \Rightarrow {}^\exists C_n \in K (C_1, \ldots , C_N) \in \mathbf{V}_k (I) \]

  \subsubsection*{例}
    $K = \mathbf{C} \ X = X_1, Y = X_2, Z = X_3 \\
      I = \langle ZX -1, X-Y \rangle \subset \mathbf C \lbrack X, Y, Z \rbrack \\
      I \cap \mathbf C \lbrack Y, Z \rbrack = \langle X - Y \rangle \\
    (C_1, C_2) \in \mathbf{V}_\mathbf{C} (I \cup \mathbf{C} \lbrack Y, Z \rbrack) = \mathbf{V}_\mathbf{C} (\langle X - Y \rangle) = \lbrace (c, c) \mid \in \mathbf{C} \rbrace $

    \paragraph{証明}
      $I ( C_1, \ldots , C_{n-1})  = \lbrace f(C_1, \ldots, C_{n-1}, X_n) \in K \lbrack X_n \rbrack \mid f \in I \rbrace$とおく \\
      明らかにこれは$K \lbrack X_n \rbrack$のイデアル 
      ($I$ がイデアル $\Leftrightarrow $
        \begin{enumerate}
          \item $I \neq \emptyset$
          \item $I \ni p, q \Rightarrow p + q \in I$
          \item $I \ni p \Rightarrow{} {}^\forall h \in K\lbrack X_1, \ldots, X_n \rbrack \  hp \in I$
        \end{enumerate})

      \subparagraph{ケース1 \\}
        $I(C_1, \ldots, C_{n-1}) = \langle 0 \rangle$ $C_n$は任意にとれる
      \subparagraph{ケース2 \\}
        $I(C_1, \ldots, C_{n-1}) = \langle f(C_1, \ldots, C_{n-1}, X_n) \rangle$
        $\deg f(C_1, \ldots, C_{n-1}, X_n) \geq 1 $ \\
        $K$は代数的閉体なので ${}^\exists C_n \in K \ f(C_1, \ldots, C_{n-1} , C_n) = 0$
      \subparagraph{ケース3 \\}
        $I(C_1, \ldots, C_{n-1}) = \langle f(C_1, \ldots, C_{n-1}, X_n) \rangle = \langle 1 \rangle$ \\
        $f(C_1, \ldots, C_{n-1}, X_n) = a$は$0$でない定数

        \[ f = S_0 (X_1, \ldots, X_{n-1} ) X_n^M + S_1(X_1, \ldots, X_{n-1})X_n^{M-1} + \cdots + S_M (X_1, \ldots, X_{n-1}) \]
        とすると
        \[S_0(C_1, \ldots, C_{n-1}) = 0, \ldots, S_{M-1} (C_1, \ldots, C_{n-1}) = 0, S_M (C_1,\ldots, C_{n-1}) = a \neq 0 \]
        \[g = t_0 (X_1, \ldots, X_{n-1})X_n^N + t_1(X_1, \ldots, X_{n-1})X_n^{N-1} + \cdots + t_N (X_1, \ldots, X_{n-1}) \]
        とする
        \[ R(g, f, X_n) \in I \cap K \lbrack X_1, \ldots, X_{n-1} \rbrack \ (\because \text{定理2.2}) \ f,g\in I\]
        \[ \det \begin{pmatrix}
            t_0 & & & s_0 & & \\
            \vdots & \ddots & & \vdots & \ddots & \\
            \vdots &  & t_0 & \vdots & & s_0 \\
            t_N & & & S_M & & \\
             & & \vdots & & & \vdots \\
              & & t_N & & S_m
          \end{pmatrix}  = h(X_1, \ldots, X_n)  \]
          矛盾、よってケース3はおこらない

\section*{3章 Hilbertの零点定理}
\subsection*{定理3.1 (HIlbertの零点定理 弱系)}
  $K$: 代数的閉体\\
  $I \subset K \lbrack X_1, \ldots, X_n \rbrack$: イデアル \\
  $\mathbf{V}_K (I) = \emptyset \Rightarrow I \ni 1$ (どんな$K$に対しても常になりたつ)

\subsection*{Claim 1}
  $f^\prime (Y_1, \ldots , Y_n, Y_{n+1}) = \\
  f(Y_1 + a_1 Y_{n+1}, \ldots, Y_n + a_n Y_{n+1}, Y_{n+1}) = h(a_1, \ldots, a_n)Y_{n+1}^N + t$ \\
  $h(Y_1, \ldots, Y_n) $は$0$でない$n$変数多項式$t$は$Y_{n+1}$に関して次数$N$未満の式と表される
\subsection*{Claim 2}
  $h(_1, \ldots, a_n) \neq 0 $なる$a_1, \ldots, a_n \in K$が存在する
\subsection*{Claim 3}
  $I^\prime = \lbrace f( Y_1 + a_1 Y_{n+1}, \ldots , Y_n + a_n Y_{n+1}, Y_{n+1}) \mid f \in I \rbrace \subset K \lbrack Y_1, \ldots, Y_{n+1} \rbrack$
  とおくと$I^\prime$はイデアルである \\

\subsection*{定理 3.2}
  $K$を無限体とするとき、\\
  $f(X_1, \ldots, X_n) \in K \lbrack X_1, \ldots , X_n \rbrack$が${}^\forall c_1, \ldots , c_n \in K \ f(c_1, \ldots, c_n) = 0 \Rightarrow f(X) = 0$
  \subsubsection*{注意1}
    $K$が有限体なら成り立たない
  \subsubsection*{注意2}
    $K$が代数的閉体なら$K$は無限体

\subsection*{定理 3.3 Hilbertの零点定理 (強形)}
  $K$: 代数的閉体 とするとき\\
  $f_1(X_1, \ldots, X_n) , \ldots, f_S(X_1, \ldots, X_n), g(X_1, \ldots, X_n) \in K \lbrack X_1, \ldots, X_n \rbrack$に対して\\
  $g\in\mathbf{I}(\mathbf{V}_K(\lbrace f_1, \ldots, f_l\rbrace)) \Rightarrow {}^\exists m \in \mathbf{N} \ g^m \in \langle f_1, \ldots, f_l \rangle$
  \subsubsection*{これの意味}
    $\mathbf{V}_K$は連立方程式
    $\begin{cases}
      f_1 = 0 \\
      \vdots \\
      f_l = 0
    \end{cases}$の$K$における解全体の集合 \\
    $g$が連立方程式のすべての解に対して$0$になるような$g$を$m$乗した$g^m$がイデアル$\langle f_1, \ldots, f_l \rangle$に属する

  \paragraph{注意}
    実は弱形は強形の特殊な場合である。

    強形を論理式のみで書くと\\
    ${}^\forall \bar{c} \in K^n \big( f_1  (\bar{c}) = 0 \wedge \cdots \wedge f_s(\bar{c}) = 0 \rightarrow f(\bar{c}) = 0\big) \rightarrow {}^\exists m \in \mathbf{N} \ f^m \in \langle f_1, \ldots , f_s \rangle \\
    \Leftrightarrow \\
    \lnot {}^\forall \bar{c} \in K^n ( \lnot \big(f_1(\bar{c}) = 0 \wedge \cdots \wedge f_s(\bar{c}) = 0\big) )\vee {}^\exists m \in \mathbf{N} \ f^m \in \langle f_1, \ldots, f_s \rangle$ \\
    $f(\bar{X})=1$に対してももちろんなりたつ、\\
    ${}^\forall f_1, \ldots, f_s \in K \lbrack \bar{X} \rbrack \\
    \lnot {}^\forall \bar{c} \in K^n \big(\lnot (f_1 (\bar{c}) = 0 \wedge \cdots \wedge f_s(\bar{c}) = 0)\big) \vee {}^\exists m \in \mathbf{N} \ 1^m \langle f_1, \ldots, f_s \rangle $

\section*{4章 連立代数方程式の解の個数}
\subsection*{定理 4.1}
  $K$: 代数的閉体 \\
  $I \subset K \lbrack X \rbrack = K \lbrack X_1, \ldots, X_n \rbrack$ に対して\\
  $\mathbf{V}_K (I)$が有限集合 $\Leftrightarrow {}^\forall  i = 1 , \ldots , n \ {}^\exists h_i (X_i) \in K \lbrack X_i \rbrack \ h_i (X_i) \in I$
\subsection*{定義 4.1}
  $I \subset K \lbrack \bar X \rbrack$: イデアル \\
  $f, g \in K \lbrack \bar X \rbrack$ に対して\\
  $f \ \sim g \overset{\text{def}}{\Leftrightarrow} f- g \in I$で$\sim$を定義すると$\sim$は同値関係になる。\\
  $\tilde{}$の同値類上に
  \begin{itemize}
    \item $\lbrack f \rbrack_{\sim} + \lbrack g \rbrack_{\sim} = \lbrack f + g \rbrack_{\sim} $
    \item $\lbrack f \rbrack_{\sim} \cdot \lbrack g \rbrack_{\sim} = \lbrack f g \rbrack_{\sim} $
  \end{itemize}
  で$+$と$\cdot$を自然に定義すると同値類は可換環になる。これを$K\lbrack \bar X \rbrack$の$I$による剰余環とよび、$K\lbrack \bar X \rbrack$で表す。 \\
  ($+$と$\cdot$がwell-definedになることは示さないといけない。\\ すなわち、$f \sim f^\prime, g \sim g^\prime$ならば、$f + g \sim f^\prime + g^\prime, f\cdot g \sim f^\prime \cdot g^\prime $を示す必要がある。)
\subsection*{定義 4.2}
  $K\lbrack \bar X \rbrack / I $は$K$上の線形空間で$I$もその部分空間とみなせる。\\
  $K \lbrack \bar X \rbrack / I $は線形空間としてのその商空間ともみなせる。\\
  その次元を$\dim K \lbrack \bar X \rbrack / I$で表す
\subsection*{定理 4.2}
  $K$: 代数的閉体 \\
  $I \subset K \lbrack \bar X \rbrack$: イデアルに対して\\
  $\mathbf{V}_K (I)$が有限集合 $\Leftrightarrow \dim K\lbrack \bar X \rbrack / I$が有限
  \subsubsection*{Claim}
    ${}^\forall f \in K \lbrack \bar{X} \rbrack \ {}^\exists f^\prime \in K \lbrack \bar{X} \rbrack \ f^\prime$ に含まれるどの多項式の$X_i$の次数$ \leq m_i$
\subsection*{定義 4.3 根基イデアル}
  イデアル$I \subset K \lbrack \bar X \rbrack$が根基であるとは \\
  ${}^\exists m \in \mathbf{N} \ f^m \in I \Rightarrow f \in I$\\
  一般のイデアル$I$に対して$I$の根基イデアル$\sqrt{I}$とは \\
  $\sqrt{I} \overset{\text{def}}{=} \lbrace f ; {}^\exists m \in \mathbf{N} \ f^m \in I \rbrace$で定義される
  \subsubsection*{補題 4.1}
    \begin{itemize}
      \item[(1)] $\sqrt{I}$はイデアル
      \item[(2)] $I$が根基 $\Leftrightarrow I = \sqrt{I}$
    \end{itemize}

\subsection*{定理 4.3}
  $K$: 代数的閉体 \\
  $I \subset K \lbrack \bar X \rbrack$: イデアル \\
  \[ \dim K \lbrack \bar X \rbrack / I < \infty \]
  に対して以下がなりたつ。
  \begin{itemize}
    \item[(1)] $\dim K \lbrack \bar X \rbrack / I \geq \mid \mathbf{V}_K (I) \mid$ 
    \item[(2)] $\dim K \lbrack \bar X \rbrack / I = \mid \mathbf{V}_K (I) \mid \Leftrightarrow I = \sqrt{I}$ 
  \end{itemize}

  \paragraph{補題 4.2}
    相異なる $\overline{c_1}, \dots, \overline{c_l} \in K^n$に対して \\
    $ h_i (\overline{c_j}) = \begin{cases}{}
        1 & i = j \\
        0 & i \neq j
      \end{cases} $なる$h_1(\bar{X}), \ldots, h_l(\bar{X}) \in K \lbrack \bar{X} \rbrack $が存在する

\subsection*{定義}
  $I \subset K \lbrack \bar{X} \rbrack$ \\
  $\dim K\lbrack\bar{X}\rbrack / I < \infty$なるイデアル \\
  前に各$X_i$に対して$0$でない1変数多項式$h_i(X_i) \in I$が存在することを示した。 \\
  このような多項式で次数が最小のものを$I$の$X_i$に関する最小多項式と呼ぶ。\\
  最小多項式は定数倍を除いて一意に定まる。

\subsection*{定理 4.4}
  $I \in K \lbrack \bar{X} \rbrack$を $\dim K \lbrack \bar X \rbrack / I < \infty$なるイデアルとし、 \\
  各$X_i$の最小多項式を$h_1(X_1), \ldots , h_l (X_l)$とおく \\
  このとき
  \[ \sqrt{I} = I + \langle h_1, \ldots, h_l \rangle \]
\end{document}
