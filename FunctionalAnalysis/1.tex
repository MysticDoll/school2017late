\documentclass[12pt,a4paper]{article}
\usepackage[truedimen,margin=25truemm]{geometry}
\usepackage{amsmath}
\usepackage[dvipdfmx]{hyperref}
\usepackage{amssymb}

\begin{document}
  \paragraph{例}
     \subparagraph{(3)}
      $\mathbf{R}$の閉区間$\lbrack a, b \rbrack$上で定義された実数値連続関数全体の集合 \\
      $C\lbrack a, b \rbrack$において、可算とスカラー倍演算を
      \begin{itemize}
        \item $(f + g) (x) = f( x) + g (x)$
        \item $(cf) (x) = c f (x)$
      \end{itemize}
      と定めると、$C\lbrack a, b \rbrack$はこれらの演算に関して線形空間をなす \\
      可算が結合則を満たすことは次のようにして確かめられる \\
      任意の$f, g, h \in C \lbrack a, b \rbrack$と$x \in \lbrack a, b \rbrack$に対して \\
      \begin{math}
        \big( (f + g) + h \big) (x) = (f + g) (x) + h(x) \\
        =  \big(f(x) + g(x) \big) + h(x) \\
        = f(x) + \big(g(x) + h(x)\big) \\
        = f(x) + (g + h) (x) \\
      = \big(f+ (g + h) \big) (x) \end{math}

      よって$(f+ g) + h = f + (g + h)$
      \setcounter{section}{3}
      \subsection{ノルム空間}
        \paragraph{定義 3.1}
        \begin{itemize}
          \item $X$: 実線形空間
          \item $\parallel \bullet \parallel : X \rightarrow \mathbf{R}$
        \end{itemize}

        このとき$\parallel \bullet \parallel$が$X$上のノルム \\
        $\Leftrightarrow$ 任意の$x, y \in X, a \in \mathbf{R}$に対して \\
        \begin{itemize}
          \item[(a)] $\parallel x \parallel \geq 0$かつ$\parallel x \parallel = 0 \Leftrightarrow x = 0$ (非負性)
          \item[(b)] $\parallel x + y \parallel \leq \parallel x \parallel + \parallel y \parallel$ (三角不等式)
          \item[(c)] $\parallel \alpha x \parallel = \mid\alpha\mid \parallel x \parallel$ (同次性)
        \end{itemize}

        また、ノルムが定義された実線形空間を{\bf ノルム空間}という
        \subparagraph{注} 
      「$X$はノルム空間」と言ったり、どのノルムを考えるのか明示したいときは「$(X, \parallel \bullet \parallel)$はノルム空間」と言ったりする

        \subparagraph{例}
          \begin{itemize}
            \item[(1)] $\parallel \bullet \parallel_2 : \mathbf{R}^N \rightarrow \mathbf{R}$を$\parallel \mathbf{x} \parallel_2 = \sqrt{x_1^2 + \cdots + x_N^2}$
              (ただし$\mathbf{x} = \begin{pmatrix} x_1 \\ \vdots \\ x_N \end{pmatrix})$と定めると
              $\parallel \bullet \parallel_2$は$\mathbf{R}^N$上のノルム
            \item[(2)] $\parallel \bullet \parallel_1 : \mathbf{R}^N \rightarrow \mathbf{R}$を$\parallel \mathbf{x} \parallel_1 = \mid x_1 \mid + \cdots  \mid x_N \mid$と定めると
              $\parallel \bullet \parallel_1$は$\mathbf{R}^N$上のノルム
            \item[(3)] $\parallel \bullet \parallel_\infty : \mathbf{R}^N \rightarrow \mathbf{R}$を$\parallel \mathbf{x} \parallel_\infty = \max_{1 \leq i \leq N} \mid x_i \mid $と定めると
              $\parallel \bullet \parallel_\infty$は$\mathbf{R}^N$上のノルム
            \item[(4)] $1 \leq p$とする。\\ $\parallel\bullet\parallel_p : \mathbf{R}^N \rightarrow \mathbf{R}$を \\
            $\parallel \mathbf{x} \parallel_p = {\big( \mid x_1\mid^p + \cdots + \mid x_N \mid^p \big)}^\frac1p$と定めると \\
              $\parallel\bullet\parallel_p$は$\mathbf{R}^N$上のノルム
          \end{itemize}

        \paragraph{注}
          ノルム空間$(X, \parallel\bullet\parallel)$に対して、$d: X^2 \rightarrow \mathbf{R}$を$d(x, y) = \parallel x - y \parallel$と定めると$d$は$X$上の距離関数

        \paragraph{証明 cf.p16}
          $x, y, z \in X$を任意に取る。
          \begin{itemize}
            \item[$(D_1)$] $d(x, y) = \parallel x - y\parallel \geq 0$ (ノルムの非負性より) \\
              $d(x, y) = 0 \Leftrightarrow \parallel x -y \parallel = 0 \Leftrightarrow x - y = 0$ (ノルムの非負性より)
            \item[$(D_2)$]  $d(y, x) = \parallel y - x \parallel = \parallel - (x - y) \parallel = \parallel (-1) (x -y) \parallel = \mid -1 \mid \parallel x - y \parallel = d(x, y)$
            \item[$(D_3)$] $d(x, z) + d(z, y) = \parallel x - z \parallel + \parallel z - y \parallel \geq \parallel (x - z) + (z - y) \parallel = \parallel x - y \parallel = d(x, y)$
          \end{itemize}

        
\end{document}

