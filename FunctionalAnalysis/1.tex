\documentclass[12pt,a4paper]{article}
\usepackage[truedimen,margin=25truemm]{geometry}
\usepackage{amsmath}
\usepackage[dvipdfmx]{hyperref}
\usepackage{pxjahyper}
\usepackage{amssymb}

\begin{document}
  \paragraph{例}
    \subparagraph{(3)}
      $\mathbf{R}$の閉区間$\lbrack a, b \rbrack$上で定義された実数値連続関数全体の集合 \\
      $C\lbrack a, b \rbrack$において、可算とスカラー倍演算を
      \begin{itemize}
        \item $(f + g) (x) = f( x) + g (x)$
        \item $(cf) (x) = c f (x)$
      \end{itemize}
      と定めると、$C\lbrack a, b \rbrack$はこれらの演算に関して線形空間をなす \\
      可算が結合則を満たすことは次のようにして確かめられる \\
      任意の$f, g, h \in C \lbrack a, b \rbrack$と$x \in \lbrack a, b \rbrack$に対して \\
      \begin{math}
        \big( (f + g) + h \big) (x) = (f + g) (x) + h(x) \\
        =  \big(f(x) + g(x) \big) + h(x) \\
        = f(x) + \big(g(x) + h(x)\big) \\
        = f(x) + (g + h) (x) \\
      = \big(f+ (g + h) \big) (x) \end{math}
      
      よって$(f+ g) + h = f + (g + h)$
  \setcounter{section}{3}
  \subsection{ノルム空間}
    \paragraph{定義 3.1}
    \begin{itemize}
      \item $X$: 実線形空間
      \item $\parallel \bullet \parallel : X \rightarrow \mathbf{R}$
    \end{itemize}
    
    このとき$\parallel \bullet \parallel$が$X$上のノルム \\
    $\Leftrightarrow$ 任意の$x, y \in X, a \in \mathbf{R}$に対して \\
    \begin{itemize}
      \item[(a)] $\parallel x \parallel \geq 0$かつ$\parallel x \parallel = 0 \Leftrightarrow x = 0$ (非負性)
      \item[(b)] $\parallel x + y \parallel \leq \parallel x \parallel + \parallel y \parallel$ (三角不等式)
      \item[(c)] $\parallel \alpha x \parallel = \mid\alpha\mid \parallel x \parallel$ (同次性)
    \end{itemize}
    
    また、ノルムが定義された実線形空間を{\bf ノルム空間}という
    \subparagraph{注} 
      「$X$はノルム空間」と言ったり、どのノルムを考えるのか明示したいときは「$(X, \parallel \bullet \parallel)$はノルム空間」と言ったりする
      
    \subparagraph{例}
      \begin{itemize}
        \item[(1)] $\parallel \bullet \parallel_2 : \mathbf{R}^N \rightarrow \mathbf{R}$を$\parallel \mathbf{x} \parallel_2 = \sqrt{x_1^2 + \cdots + x_N^2}$
          (ただし$\mathbf{x} = \begin{pmatrix} x_1 \\ \vdots \\ x_N \end{pmatrix})$と定めると
          $\parallel \bullet \parallel_2$は$\mathbf{R}^N$上のノルム
        \item[(2)] $\parallel \bullet \parallel_1 : \mathbf{R}^N \rightarrow \mathbf{R}$を$\parallel \mathbf{x} \parallel_1 = \mid x_1 \mid + \cdots  \mid x_N \mid$と定めると
          $\parallel \bullet \parallel_1$は$\mathbf{R}^N$上のノルム
        \item[(3)] $\parallel \bullet \parallel_\infty : \mathbf{R}^N \rightarrow \mathbf{R}$を$\parallel \mathbf{x} \parallel_\infty = \max_{1 \leq i \leq N} \mid x_i \mid $と定めると
          $\parallel \bullet \parallel_\infty$は$\mathbf{R}^N$上のノルム
        \item[(4)] $1 \leq p$とする。\\ $\parallel\bullet\parallel_p : \mathbf{R}^N \rightarrow \mathbf{R}$を \\
        $\parallel \mathbf{x} \parallel_p = {\big( \mid x_1\mid^p + \cdots + \mid x_N \mid^p \big)}^\frac1p$と定めると \\
          $\parallel\bullet\parallel_p$は$\mathbf{R}^N$上のノルム
      \end{itemize}
    
    \paragraph{注}
      ノルム空間$(X, \parallel\bullet\parallel)$に対して、$d: X^2 \rightarrow \mathbf{R}$を$d(x, y) = \parallel x - y \parallel$と定めると$d$は$X$上の距離関数
    
    \paragraph{証明 cf.p16}
      $x, y, z \in X$を任意に取る。
      \begin{itemize}
        \item[$(D_1)$] $d(x, y) = \parallel x - y\parallel \geq 0$ (ノルムの非負性より) \\
          $d(x, y) = 0 \Leftrightarrow \parallel x -y \parallel = 0 \Leftrightarrow x - y = 0$ (ノルムの非負性より)
        \item[$(D_2)$]  $d(y, x) = \parallel y - x \parallel = \parallel - (x - y) \parallel = \parallel (-1) (x -y) \parallel = \mid -1 \mid \parallel x - y \parallel = d(x, y)$
        \item[$(D_3)$] $d(x, z) + d(z, y) = \parallel x - z \parallel + \parallel z - y \parallel \geq \parallel (x - z) + (z - y) \parallel = \parallel x - y \parallel = d(x, y)$
      \end{itemize}
    
    \paragraph{注}
      $X$が実線形空間をなし、$d$が$X$上の距離であるとき
      $\parallel \bullet \parallel : X \rightarrow \mathbf{R}$を
      \[\parallel \mathbf{x} \parallel = d(\mathbf{x}, \mathbf{0}) \]
      と定義する。
    
      このとき、$d$が次の条件を満たすならば
      $\parallel \bullet \parallel$は$X$上のノルムになる \\
      任意の$x, y, z \in X$と$\alpha \in \mathbf{R}$に対して
      \begin{enumerate}
        \item 
      \end{enumerate}
      
    \paragraph{性質 3.1}
      $(X, \parallel \bullet \parallel)$: ノルム空間 \\
      任意の$x,y \in X$に対して
      \[ \mid \parallel x \parallel - \parallel y \parallel \mid \leq \parallel x - y\parallel \]
    \subparagraph{証明}
      \[ \parallel x \parallel = \parallel x - y + y \parallel \leq \parallel x - y \parallel + \parallel y \parallel \]
      よって$\parallel x \parallel - \parallel y \parallel \leq \parallel x - y \parallel$\\
      \[\parallel y \parallel = \parallel y - x + x \parallel \leq \parallel y - x \parallel + \parallel x \parallel \]
      よって$\parallel y \parallel - \parallel x \parallel \leq \parallel y - x \parallel$
      
    \paragraph{定義 3.2 \\}
      \noindent$X$: ノルム空間 \\
      ${(x_n)}_{n=1}^\infty$: $X$の点列 \\
      $x \in X$ \\
      このとき、${(x_n)}_{n=1}^\infty$が収束$\Leftrightarrow \parallel x_n - x \parallel \rightarrow 0 \ (n \rightarrow 0)$
    
    \paragraph{定義 3.3 (写像の連続)}
      (cf.\ 定義2.8 p39)
    \paragraph{注意1}
      ノルムは連続\\
      \raise0.2ex\hbox{\textcircled{\scriptsize{$\because$}}}
      $X$の点列${(x_n)}_{n=1}^\infty$と$x \in X$について$x_n \rightarrow x \ (x \rightarrow \infty)$であるとき、
      \[ 0 \leq \mid \parallel x_n \parallel - \parallel x \parallel \mid \leq \parallel x_n -  x\parallel \rightarrow 0 \ (n \rightarrow \infty) \]
    \paragraph{定義 3.4 \\}
      $X$: 実線形空間 \\
      $\parallel \bullet \parallel_a, \parallel \bullet \parallel_b$ $X$上のノルム 
    
      $\parallel \bullet \parallel_a$と$\parallel\bullet\parallel_b$が等価 \\
      $\Leftrightarrow$ 任意の$x \in X$にs対して、
      \[ M_1 \parallel x \parallel_a \leq \parallel x \parallel_b \leq M_2 \parallel x \parallel_a \]
      をみたす$M_1, M_2 > 0$が存在
    \paragraph{注意2}
      (a) (b) ノルムが等価であるという関係は同値関係: \\
      $X$上の任意の$\parallel\bullet\parallel_a, \parallel\bullet\parallel_b, \parallel\bullet\parallel_c$に対して、
      \begin{itemize}
        \item[反射律] $\parallel\bullet\parallel_a$と$\parallel\bullet\parallel_b$は等価
        \item[対象律] $\parallel\bullet\parallel_a$と$\parallel\bullet\parallel_b$が等価 $\Leftrightarrow$ $\parallel\bullet\parallel_b$と$\parallel\bullet\parallel_a$は等価
        \item[推移律] $\parallel\bullet\parallel_a$と$\parallel\bullet\parallel_b$が等価かつ$\parallel\bullet\parallel_b$と$\parallel\bullet\parallel_c$が等価 $\Rightarrow$ $\parallel\bullet\parallel_a$と$\parallel\bullet\parallel_c$は等価
      \end{itemize}
      (c) 点列$x_n \in X$と$x \in X$に対して$X$上のノルム$\parallel\bullet\parallel_a$と$\parallel\bullet\parallel_b$が等価ならば\\
      ${(x_n)}_{n=1}^\infty$が$\parallel\bullet\parallel_b$の意味で$x$に収束することは
      ${(x_n)}_{n=1}^\infty$が$\parallel\bullet\parallel_a$の意味で$x$に収束することと同値である。
    \paragraph{定理3.1}
      $X$を有限次元ベクトル空間とする。\\
      このとき、$X$上の全てのノルムは等価
    \paragraph{証明}
    (a) $\dim X = N$として$X$の基底の1つを$\lbrace u_1, u_2, \ldots, u_n \rbrace$とする。\\
      このとき$x \in X$に対して$\parallel x \parallel_2 = \sqrt{x_1^2 + \cdots + x_n^2}$\\
      ただし$x = x_1 u_1 + \cdots + x_N u_n$
      と定義すると\\
      $\parallel \bullet \parallel_2 : X \rightarrow \mathbf{R}$は$X$上のノルム
    
    (b) $X$上の任意のノルムを$J(\bullet)$とする
    
    (i) 任意の$x \in X$に対して\\
    $x = x_1 u_1 + \cdots + x_N u_N$とすると\\
    $J(x) = J(x_1 u_1 + \cdots + x_N u_N) \\
     \leq J(X_1 u_1) + \cdots + J(x_N u_N) \\
     = \mid x_1 \mid J(u_1) + \cdots + \mid x_n \mid J(u_N) \\
     \leq \sqrt{\mid x_1 \mid^2 + \cdots + \mid x_N \mid^2} \sqrt{J{(u_1)}^2 + \cdots + J{(u_N)}^2}$ \\
     従って $M_2 = \sqrt{J{(u_1)}^2 + \cdots + J{(u_N)}^2}$とおくと \\
     $J(x)\leq M_2 \parallel x \parallel_2$が成り立つ
    
     (ii) 任意の$x, y \in X$に対して\\
     $ 0 \leq \mid J(x) - J(y) \mid \\ \leq J(x - y) \\ \leq M_2 \parallel x - y \parallel_2$ \\
     よって$x \rightarrow y \Rightarrow J(x) \rightarrow J(y)$であるので \\ $J: X \rightarrow \mathbf{R}$はノルム空間$(X, \parallel\bullet\parallel_2)$上の連続関数\\
     ここで、$\tilde{J}: \mathbf{R} \rightarrow \mathbf{R}$を\\
    $\tilde{J}(x_1, \ldots ,x_N) = J(x_1 u_1 + \cdots + x_N u_N)$と定義すると$\tilde{J}$は$(\mathbf{R}^N, d_2)$上の連続関数
    
    (iii) $S = \lbrace x \in X \mid \parallel x \parallel_2 =1 \rbrace \\ \tilde{S} = \lbrace (x_1, \ldots, x_N) \in \mathbf{R}^N \mid \sqrt{x_1^2 + \cdots + x_N^2} = 1 \rbrace$とすると \\
    $\tilde{S}$は$(\mathbf{R}^N, d_2)$における有界閉集合であるので、定理2.3 (b) (Heine-Borelの被覆定理)よりコンパクト集合。\\
    よって定理2.5より、$\tilde{J}$は$\tilde{S}$上で最小値$M_1$をもつ、従って$J$は$S$上で最小値$M_1$を持つ。\\
    $x\in S$のとき、$x \neq 0$であるので$J(x) > 0$よって$M_1 > 0$
    
    以上から
    \begin{itemize}
      \item $x \neq 0$のとき \\
      $M_1 \leq J (\frac{1}{\parallel x \parallel_2} x) = \frac{1}{\parallel x \parallel_2} x$ \\
        従って$M_1\parallel x \parallel_2 \leq J(x)$
      \item $x = 0$のとき \\ $M_1 \parallel x\parallel_2 = 0 = J(x)$
    \end{itemize}
  \paragraph{定義 3.5}
    $X$: ノルム空間
    \begin{itemize}
      \item[(a)] 部分空間$M \subset X$が閉集合であるとき、これを閉部分空間であるという。
        \begin{itemize}
          \item 部分空間$M \subset X \Rightarrow \bar{M}$ (閉包)は閉部分空間 
          \item ${}^\forall S \subseteq X$, \text{span}{(S)}は閉部分空間
        \end{itemize}
      \item[(b)] $v \in X$ \\ $M$: $X$の閉部分空間 \\ このとき $V = v + M \overset{\text{def.}}{=} \lbrace v + m \mid m \in M  \rbrace$を線形多様体という
      \item[(c)] $C \subseteq X$に対して、\\ $C$が凸集合 $\Leftrightarrow$ ${}^\forall x_1, x_2 \in C, {}^\forall \lambda \in \lbrack 0, 1\rbrack, \ \lambda x_1 + (1 - \lambda) x_2 \in C$
    \end{itemize}

  \paragraph{性質 3.2}
    $X$をノルム空間とする
    \begin{itemize}
      \item[(a)] $v_0 \in X$ \\ $M$: $X$の閉部分空間 \\ $V = v_0 + M$ (線形多様体) \\ このとき${}^\forall v \in V, \ V = v + M$ \\ 線形多様体は閉凸集合。
      \item[(b)] $M_\alpha$: $X$の閉部分空間 ($\alpha \in \mathcal{M}$) \\ $M = \underset{\alpha \in \mathcal{M}}{\cap} M_\alpha (\neq \emptyset) \Rightarrow M$は閉部分空間 \\
                 $C_\alpha$: $X$の凸部分集合 ($\alpha \in \mathcal{M}$) \\ $C = \underset{\alpha \in \mathcal{M}}{\cap} C_\alpha (\neq \emptyset) \Rightarrow C$は閉凸集合 \\
      \item[(c)] 略
      \item[(d)] $X$の部分空間$M$の閉包$\bar{M}$は部分空間になる。また、凸集合$C \subset X$の閉包$\bar{C}$も凸集合になる。
    \end{itemize}
  \paragraph{定理 3.2}
    $X$: ノルム空間, $M$: $X$の有限次元部分空間 $\Rightarrow M$は閉集合
  \subsection{内積空間}
  \paragraph{定義 3.6}
    実 (または複素)線形空間$X$において次の (I-a) から (I-c)を満たす写像
    \[ \langle \bullet, \bullet \rangle : X^2 \rightarrow \mathbf{R} \ (\text{または} \  \mathbf{C}) \]
    を$X$上の内積といい、また、$x,y\in X$に対して$\langle x,y \rangle$を$x$と$y$の内積という。このとき$X$を内積空間という

    ${}^\forall x, y, z \in X$と$\alpha, \beta \in \mathbf{R} \ (\text{または} \ \mathbf{C})$
    \begin{itemize}
      \item[(I-a)] 対称性 (またはエルミート性): $\langle x, y \rangle = \overline{\langle y, x \rangle}$
      \item[(I-b)] 非負性: $\langle x, x \rangle \geq 0$ \\ $\langle x, x \rangle = 0 \Leftrightarrow x = 0$
      \item[(I-c)] 線形性: $\langle \alpha x + \beta y, z \rangle = \alpha \langle x, z \rangle + \beta \langle y, z \rangle$
    \end{itemize}

    さらに、$x,y\in X$に対して
    \[ x \perp y \Leftrightarrow \langle x, y \rangle = 0 \]
    を$x$と$y$が直交するといい。\\
    そして$\emptyset \neq S \subseteq X$に対して
    \[S^\perp = \lbrace x \in X \mid {}^\forall y \in S, x \perp y \rbrace \]
    と表すことにする

    \subparagraph{注意}
      ${}^\forall x,y \in X$に対して\\
      \[x\perp y \Leftrightarrow y \perp x\]
    \subparagraph{注意}
      ${}^\forall x \in X$に対して
      \[\langle x, \rangle = 0 , \therefore x \perp 0 \]
    \subparagraph{注意}
      ${}^\forall S \subseteq X ( S \neq \emptyset)$に対して、$S^\perp$は$X$の部分空間

    \subparagraph{例}
      $\mathbf{R}^n$において\\
      \[ \langle \mathbf{x}, \mathbf{y} \rangle = x_1 y_1 + x_2 y_2 + \cdots + x_n y_n\]
      (ただし$\mathbf{x} = \begin{pmatrix}x_1 \\ \vdots \\ x_n\end{pmatrix}, \mathbf{y} = \begin{pmatrix}y_1 \\ \vdots \\ y_n \end{pmatrix} \in \mathbf{R}^n$)\\
      は$\mathbf{R}^n$上の内積を定める。

    \subparagraph{例}
      $M_{m,n}$ ($m \times n$の実行列全体において) 
      \[ \langle A, B \rangle = \underset{i=1}{ \overset{m}{\sum}}\underset{j=1}{\overset{n}{\sum}} a_{ij} b_{ij} = \text{tr} ({}^t BA)\]
      (ただし$A= (a_{ij}), B = (b_{ij}) \in M_{m,n}$)は$_{m,n}$上の内積を定める。  (Hilbert-Schmidt内積)
    \subparagraph{例}
      $C\lbrack0,1\rbrack = $ (閉区間$\lbrack0,1\rbrack (\subseteq \mathbf{R})$上の実数値連続関数全体の集合) 
      \[\langle f, g \rangle = \int_0^1 f(x) g(x) \text{dx} \]
      は$C\lbrack0,1\rbrack$上の内積を定める
  \paragraph{性質 3.3}
    $(X, \langle \bullet, \bullet \rangle)$を内積空間とする。\\
    このとき、写像$\parallel\bullet\parallel: X \rightarrow \mathbf{R}$を
    \[ \parallel x\parallel = \sqrt{\langle x, x \rangle} \]
    と定義すると、$\parallel\bullet\parallel$は$X$上のノルムとなる。\\
    このノルムを内積から誘導されたノルムといい、次が成立

    任意の$x,y\in X$に対して
    \begin{itemize}
      \item[(a)] (Chauchy-Schwarzの不等式) $\mid\langle x, y \rangle\mid \leq \parallel x\parallel \parallel y\parallel$ 等号成立 $\Leftrightarrow x, y$が線形従属
      \item[(b)] $\parallel x \parallel = \sup_{\parallel u \parallel = 1} \mid \langle x, u \rangle \mid = \max_{\parallel u \parallel = 1} \mid \langle x, u\rangle \mid$
      \item[(c)] (三角不等式) $\parallel x + y \parallel \leq \parallel x\parallel + \parallel y \parallel$ 等号成立 $\Leftrightarrow x, y$の一方が他方の非負実数倍
    \end{itemize}
  \paragraph{補題 3.1}
  $(X, \langle \bullet,\bullet \rangle)$: 内積空間 
  \begin{itemize}
    \item[(a)] 内積の連続性 \\
      ${(x_n)}_{n=1}^\infty , {(y_n)}_{n=1}^\infty$: $X$の点列 \\
      $x, y \in X$ このとき \\
      $x_n \rightarrow x, y_n \rightarrow y \Rightarrow \langle x_n, y_n \rangle \rightarrow \langle x, y \rangle $
    \item[(b)] $S \subset X$に対して \\
      $S^\perp = \lbrace x \in X \mid x \perp y \ {}^\forall y \in S \rbrace $
      は閉部分空間
    \end{itemize}
  
  \paragraph{性質 3.3 (a)}
    内積空間$X$において \\
    \[ \parallel x + y \parallel^2 + \parallel x - y \parallel^2 = 2(\parallel x \parallel^2 + \parallel y \parallel^2) \]
    ただし$\parallel\bullet\parallel$は内積から誘導されたノルム

  \paragraph{Jordan-Neumannの定理}
    \begin{itemize}
      \item[(1)] 実ノルム空間$(X, \parallel\bullet\parallel)$において中線定理が成立しているならば \\
        $\langle\bullet,\bullet\rangle: X \rightarrow \mathbf{R}$を$\langle x, y \rangle = \frac14 ( \parallel x + y \parallel^2 - \parallel x - y \parallel^2 )$で定義すると \\
        $(X, \langle\bullet,\bullet\rangle)$は実内積空間となり、$\parallel\bullet\parallel$はこの内積から誘導されたノルムとなる
    \end{itemize}

  \paragraph{定理 3.3}
    \begin{itemize}
      \item[(1)] 実ノルム空間$(X, \parallel\bullet\parallel)$が中線定理を満たすとき、\\
        $\langle x, y \rangle = \frac 14 ( \parallel x + y \parallel^2 - \parallel x - y \parallel^2 )$ \\
        により写像$\langle \bullet,\bullet \rangle : X^2 \rightarrow \mathbf{R}$を定義すると \\
        $\langle\bullet,\bullet\rangle$は$X$上の内積であり、$\parallel\bullet\parallel$は$\langle\bullet,\bullet\rangle$から誘導されたノルムである。
      \item[(2)] 複素ノルム空間$(X, \parallel\bullet\parallel)$のノルム$\parallel\bullet\parallel$が中線定理を満たすとき \\
        $\langle x, y \rangle = \frac 14 (\parallel x + y \parallel^2 - \parallel x - y \parallel + i \parallel x + i y \parallel^2 - i \parallel x - i y \parallel^2)$ \\
        を定義すると$\langle \bullet,\bullet\rangle$は内積の公理を満たし、$\parallel x \parallel = \sqrt{\langle x, x \rangle} \ {}^\forall x \in X$ を満たす
    \end{itemize}
  
  \paragraph{補題 3.2}
    連続関数$f: \mathbf{R} \rightarrow \mathbf{R}$が \\
    $f(s+t) = f(s) + f(t) \ {}^\forall s, t \in \mathbf{R}$ \\
    をみたすとき$f$は定数$c \in \mathbf{R}$を用いて \\
    $f(t) = ct \ (t \in \mathbf{R}) $ \\
    と表せる
  
  \paragraph{定義 3.7}
    \begin{itemize}
      \item[(a)] $u \subseteq X$について
        \begin{itemize}
          \item $u$が直交系$\Leftrightarrow$ $0 \not{\in} u$かつ$x,y \in u, \ x \neq y \Rightarrow x \perp y$
          \item $u$が正規直交系$\Leftrightarrow$A $u$は直交系かつ任意の$x \in u$に対して$\parallel x \parallel = 1$
        \end{itemize}
      \item[(b)] $u \in X$について
        \begin{itemize}
          \item $u$が極大 $\Leftrightarrow$ 順序集合$(s^X, \subseteq)$において、$u$は極大 \\
            $\Leftrightarrow$ $u \subsetneq u^\prime$なる直交系$u^\prime$が存在しない \\
            $\Leftrightarrow$ 任意の$v \in X \backslash u$に対して$u \cup \lbrace v \rbrace$は直交系でない
        \end{itemize}
    \end{itemize}

  \paragraph{性質 3.4}
    $X$を内積空間とする
    \begin{itemize}
      \item[(a)] $v \subseteq X$について$v$が直交系$\Rightarrow$ $v$は線形独立 
      \item[(b)] Gram-Schmidtの直交化法
      \item[(c)] Besselの不等式 \\
        正規直交系$\lbrace u_j \rbrace_{j=1}^\infty$が与えられるとき \\
        任意の$x \in X$に対して $\sum_{j=1}^\infty \mid \langle x, u_j \rangle \mid^2 \leq \parallel x \parallel^2$
    \end{itemize}

  \subsection{ノルム空間の有界線形作用素}
    \paragraph{定義 3.8}
      $(X, \parallel \bullet\parallel_X), (Y, \parallel\bullet\parallel_Y)$ ノルム空間 \\
      \begin{itemize}
        \item[(a)] 線形写像$A : X \rightarrow Y$について、 \\
          $A$は有界 $\Leftrightarrow$ 「任意の$x \in X$に対して \\
            $\parallel A(x) \parallel_Y \leq K \parallel x \parallel_X$ \\
          を満たすような$K \geq 0$が存在」 \\
          有界な線形写像を 有界線形作用素という \\
          $X$から$Y$への有界線形作用素全体の集合を \\
          $B(X, Y)$と表す。特に$B(X, X)$は$B(X)$と表す
        \item[(b)] 有界線形作用素 $A : X \rightarrow Y$に対して \\
          $\parallel A \parallel = \inf \lbrace K \geq 0 \mid \parallel A(x) \parallel_Y \leq K \parallel x \parallel_X \ {}^\forall x \in X \rbrace \\
          = \underset{x \neq 0}{\sup} \frac{\parallel A(x) \parallel_Y}{\parallel x \parallel_X}
          = \underset{x \neq 0}{\sup}\parallel \frac{1}{\parallel x \parallel_X} A(x) \parallel_Y = \underset{x \neq 0}{\sup}\parallel A (\frac{1}{\parallel x \parallel_X} x) \parallel_Y \\
          = \underset{\parallel x \parallel_X = 1}{\sup} \parallel A(x) \parallel_Y = \underset{x \leq 1}{\sup} \parallel A (x) \parallel_Y$ \\
          を作用素ノルムと呼ぶ。
        \item[(c)] $X$から$\mathbf{R}$への有界線形写像を有界線形汎関数という。
      \end{itemize}
    \subparagraph{例題 3.2}
      $(X, \parallel \bullet \parallel_X), (Y, \parallel \bullet \parallel_Y), (Z, \parallel \bullet \parallel_Z)$: ノルム空間\\
      $A_1 : Y \rightarrow Z$ 有界線形作用素 \\
      $A_2 : X \rightarrow Y$ 有界線形作用素 \\
      このとき$A_1, A_2$の合成 \\
      $A_1 A_2 (x): X \rightarrow Z$は$X$から$Z$への有界線形作用素で \\
      $\parallel A_1 A_2 \parallel \leq \parallel A_1 \parallel \parallel A_2 \parallel $ \\
      (作用素ノルムの劣乗法性)

    \paragraph{定理 3.4}
      $(X, \parallel \bullet\parallel_X), (Y, \parallel \bullet\parallel_Y)$: ノルム空間 \\
      $A : X \rightarrow Y$: 線形作用素 \\
      このとき、$A$は連続 $\Leftrightarrow$ $A$ は有界

    \paragraph{定理 3.5}
      有界線形写像の逆写像はまた線形写像となる

    \paragraph{定理 3.6}
      $(X, \parallel\bullet\parallel_X), (Y, \parallel\bullet\parallel_Y)$: ノルム空間 \\
      $B(X, Y)$ $X$から$Y$への有界線形作用素全体の集合 \\
      このとき$A_1, A_2 \in B(X, Y), \ \alpha \in \mathbf{R}$に対して \\
      $A_1 + A_2, \alpha A_1$を \\
      $(A_1 + A_2) (x) = A_1 (x) + A_2 (x) \\
      (\alpha X_1)(x) = \alpha (A_1 (x)) $により定義される$X$から$Y$への写像とすると \\
      $B(X, Y)$はこれらの演算に関して線形空間をなし、作用素ノルム$\parallel\bullet\parallel$は$B(X, Y)$上のノルムである

  \subsection{行列とベクトルのノルム}
    \paragraph{定理 3.7}
    $(A=a_{jk})$ $m\times n$行列 \\
    $X = \mathbf{R}^n, Y= \mathbf{R}^m$ \\
    ここで、写像$T : X \rightarrow Y$を$T(\mathbf{x}) = A \mathbf{x} \ (\mathbf{X} \in \mathbf{R}^m)$と定めると、$T$は線形となる \\

    \begin{itemize}
      \item[(a)] ノルム空間$(X, \parallel\bullet\parallel_2), (Y, \parallel\bullet\parallel_2)$に対して$T$は有界
      \item[(b)] $X, Y$に定義されるいかなるノルムに対しても$T$は有界
    \end{itemize}

    \subparagraph{例題 3.3}
      $m\times n$行列$A= (a_{jk})$に対して、$A$によって定義される$\mathbf{R}^n$から$\mathbf{R}^m$への線形作用素のノルムについて

      \begin{itemize}
        \item[(a)] $\mathbf{R}^n, \mathbf{R}^m$のノルムをともに$\parallel\bullet\parallel_1$とするとき、 \\
          $\parallel A \parallel_1 = \underset{k}{\max} \underset{j=1}{\overset{m}{\sum}}\mid a_{jk} \mid$
        \item[(b)] $\mathbf{R}^n, \mathbf{R}^m$のノルムをともに$\parallel\bullet\parallel_\infty$とするとき \\
          $\parallel A \parallel_\infty = \underset{j}{\max} \underset{k=1}{\overset{n}{\sum}} \mid a_{jk} \mid$
      \end{itemize}
\end{document}
