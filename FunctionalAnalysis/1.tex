\documentclass[12pt,a4paper]{article}
\usepackage[truedimen,margin=25truemm]{geometry}
\usepackage{amsmath}
\usepackage[dvipdfmx]{hyperref}
\usepackage{amssymb}

\begin{document}
  \paragraph{例}
     \subparagraph{(3)}
      $\mathbf{R}$の閉区間$\lbrack a, b \rbrack$上で定義された実数値連続関数全体の集合 \\
      $C\lbrack a, b \rbrack$において、可算とスカラー倍演算を
      \begin{itemize}
        \item $(f + g) (x) = f( x) + g (x)$
        \item $(cf) (x) = c f (x)$
      \end{itemize}
      と定めると、$C\lbrack a, b \rbrack$はこれらの演算に関して線形空間をなす \\
      可算が結合則を満たすことは次のようにして確かめられる \\
      任意の$f, g, h \in C \lbrack a, b \rbrack$と$x \in \lbrack a, b \rbrack$に対して \\
      \begin{math}
        \big( (f + g) + h \big) (x) = (f + g) (x) + h(x) \\
        =  \big(f(x) + g(x) \big) + h(x) \\
        = f(x) + \big(g(x) + h(x)\big) \\
        = f(x) + (g + h) (x) \\
      = \big(f+ (g + h) \big) (x) \end{math}

      よって$(f+ g) + h = f + (g + h)$
      \setcounter{section}{3}
      \subsection{ノルム空間}
        \paragraph{定義 3.1}
        \begin{itemize}
          \item $X$: 実線形空間
          \item $\parallel \bullet \parallel : X \rightarrow \mathbf{R}$
        \end{itemize}

        このとき$\parallel \bullet \parallel$が$X$上のノルム \\
        $\Leftrightarrow$ 任意の$x, y \in X, a \in \mathbf{R}$に対して \\
        \begin{itemize}
          \item[(a)] $\parallel x \parallel \geq 0$かつ$\parallel x \parallel = 0 \Leftrightarrow x = 0$ (非負性)
          \item[(b)] $\parallel x + y \parallel \leq \parallel x \parallel + \parallel y \parallel$ (三角不等式)
          \item[(c)] $\parallel \alpha x \parallel = \mid\alpha\mid \parallel x \parallel$ (同次性)
        \end{itemize}

        また、ノルムが定義された実線形空間を{\bf ノルム空間}という
        \subparagraph{注} 
      「$X$はノルム空間」と言ったり、どのノルムを考えるのか明示したいときは「$(X, \parallel \bullet \parallel)$はノルム空間」と言ったりする

        \subparagraph{例}
          \begin{itemize}
            \item[(1)] $\parallel \bullet \parallel_2 : \mathbf{R}^N \rightarrow \mathbf{R}$を$\parallel \mathbf{x} \parallel_2 = \sqrt{x_1^2 + \cdots + x_N^2}$
              (ただし$\mathbf{x} = \begin{pmatrix} x_1 \\ \vdots \\ x_N \end{pmatrix})$と定めると
              $\parallel \bullet \parallel_2$は$\mathbf{R}^N$上のノルム
            \item[(2)] $\parallel \bullet \parallel_1 : \mathbf{R}^N \rightarrow \mathbf{R}$を$\parallel \mathbf{x} \parallel_1 = \mid x_1 \mid + \cdots  \mid x_N \mid$と定めると
              $\parallel \bullet \parallel_1$は$\mathbf{R}^N$上のノルム
            \item[(3)] $\parallel \bullet \parallel_\infty : \mathbf{R}^N \rightarrow \mathbf{R}$を$\parallel \mathbf{x} \parallel_\infty = \max_{1 \leq i \leq N} \mid x_i \mid $と定めると
              $\parallel \bullet \parallel_\infty$は$\mathbf{R}^N$上のノルム
            \item[(4)] $1 \leq p$とする。\\ $\parallel\bullet\parallel_p : \mathbf{R}^N \rightarrow \mathbf{R}$を \\
            $\parallel \mathbf{x} \parallel_p = {\big( \mid x_1\mid^p + \cdots + \mid x_N \mid^p \big)}^\frac1p$と定めると \\
              $\parallel\bullet\parallel_p$は$\mathbf{R}^N$上のノルム
          \end{itemize}

        \paragraph{注}
          ノルム空間$(X, \parallel\bullet\parallel)$に対して、$d: X^2 \rightarrow \mathbf{R}$を$d(x, y) = \parallel x - y \parallel$と定めると$d$は$X$上の距離関数

        \paragraph{証明 cf.p16}
          $x, y, z \in X$を任意に取る。
          \begin{itemize}
            \item[$(D_1)$] $d(x, y) = \parallel x - y\parallel \geq 0$ (ノルムの非負性より) \\
              $d(x, y) = 0 \Leftrightarrow \parallel x -y \parallel = 0 \Leftrightarrow x - y = 0$ (ノルムの非負性より)
            \item[$(D_2)$]  $d(y, x) = \parallel y - x \parallel = \parallel - (x - y) \parallel = \parallel (-1) (x -y) \parallel = \mid -1 \mid \parallel x - y \parallel = d(x, y)$
            \item[$(D_3)$] $d(x, z) + d(z, y) = \parallel x - z \parallel + \parallel z - y \parallel \geq \parallel (x - z) + (z - y) \parallel = \parallel x - y \parallel = d(x, y)$
          \end{itemize}

        \paragraph{注}
          $X$が実線形空間をなし、$d$が$X$上の距離であるとき
          $\parallel \bullet \parallel : X \rightarrow \mathbf{R}$を
          \[\parallel \mathbf{x} \parallel = d(\mathbf{x}, \mathbf{0}) \]
          と定義する。

          このとき、$d$が次の条件を満たすならば
          $\parallel \bullet \parallel$は$X$上のノルムになる \\
          任意の$x, y, z \in X$と$\alpha \in \mathbf{R}$に対して
          \begin{enumerate}
            \item 
          \end{enumerate}

      \paragraph{性質 3.1}
        $(X, \parallel \bullet \parallel)$: ノルム空間 \\
        任意の$x,y \in X$に対して
        \[ \mid \parallel x \parallel - \parallel y \parallel \mid \leq \parallel x - y\parallel \]
      \subparagraph{証明}
        \[ \parallel x \parallel = \parallel x - y + y \parallel \leq \parallel x - y \parallel + \parallel y \parallel \]
        よって$\parallel x \parallel - \parallel y \parallel \leq \parallel x - y \parallel$\\
        \[\parallel y \parallel = \parallel y - x + x \parallel \leq \parallel y - x \parallel + \parallel x \parallel \]
        よって$\parallel y \parallel - \parallel x \parallel \leq \parallel y - x \parallel$

      \paragraph{定義 3.2 \\}
        \noindent$X$: ノルム空間 \\
        ${(x_n)}_{n=1}^\infty$: $X$の点列 \\
        $x \in X$ \\
        このとき、${(x_n)}_{n=1}^\infty$が収束$\Leftrightarrow \parallel x_n - x \parallel \rightarrow 0 \ (n \rightarrow 0)$

      \paragraph{定義 3.3 (写像の連続)}
        (cf.\ 定義2.8 p39)
      \paragraph{注意1}
        ノルムは連続\\
        \raise0.2ex\hbox{\textcircled{\scriptsize{$\because$}}}
        $X$の点列${(x_n)}_{n=1}^\infty$と$x \in X$について$x_n \rightarrow x \ (x \rightarrow \infty)$であるとき、
        \[ 0 \leq \mid \parallel x_n \parallel - \parallel x \parallel \mid \leq \parallel x_n -  x\parallel \rightarrow 0 \ (n \rightarrow \infty) \]
      \paragraph{定義 3.4 \\}
        $X$: 実線形空間 \\
        $\parallel \bullet \parallel_a, \parallel \bullet \parallel_b$ $X$上のノルム 

        $\parallel \bullet \parallel_a$と$\parallel\bullet\parallel_b$が等価 \\
        $\Leftrightarrow$ 任意の$x \in X$にs対して、
        \[ M_1 \parallel x \parallel_a \leq \parallel x \parallel_b \leq M_2 \parallel x \parallel_a \]
        をみたす$M_1, M_2 > 0$が存在
      \paragraph{注意2}
        (a) (b) ノルムが等価であるという関係は同値関係: \\
        $X$上の任意の$\parallel\bullet\parallel_a, \parallel\bullet\parallel_b, \parallel\bullet\parallel_c$に対して、
        \begin{itemize}
          \item[反射律] $\parallel\bullet\parallel_a$と$\parallel\bullet\parallel_b$は等価
          \item[対象律] $\parallel\bullet\parallel_a$と$\parallel\bullet\parallel_b$が等価 $\Leftrightarrow$ $\parallel\bullet\parallel_b$と$\parallel\bullet\parallel_a$は等価
          \item[推移律] $\parallel\bullet\parallel_a$と$\parallel\bullet\parallel_b$が等価かつ$\parallel\bullet\parallel_b$と$\parallel\bullet\parallel_c$が等価 $\Rightarrow$ $\parallel\bullet\parallel_a$と$\parallel\bullet\parallel_c$は等価
        \end{itemize}
        (c) 点列$x_n \in X$と$x \in X$に対して$X$上のノルム$\parallel\bullet\parallel_a$と$\parallel\bullet\parallel_b$が等価ならば\\
        ${(x_n)}_{n=1}^\infty$が$\parallel\bullet\parallel_b$の意味で$x$に収束することは
        ${(x_n)}_{n=1}^\infty$が$\parallel\bullet\parallel_a$の意味で$x$に収束することと同値である。
      \paragraph{定理3.1}
        $X$を有限次元ベクトル空間とする。\\
        このとき、$X$上の全てのノルムは等価
      \paragraph{証明}
      (a) $\dim X = N$として$X$の基底の1つを$\lbrace u_1, u_2, \ldots, u_n \rbrace$とする。\\
        このとき$x \in X$に対して$\parallel x \parallel_2 = \sqrt{x_1^2 + \cdots + x_n^2}$\\
        ただし$x = x_1 u_1 + \cdots + x_N u_n$
        と定義すると\\
        $\parallel \bullet \parallel_2 : X \rightarrow \mathbf{R}$は$X$上のノルム

      (b) $X$上の任意のノルムを$J(\bullet)$とする

      (i) 任意の$x \in X$に対して\\
      $x = x_1 u_1 + \cdots + x_N u_N$とすると\\
      $J(x) = J(x_1 u_1 + \cdots + x_N u_N) \\
       \leq J(X_1 u_1) + \cdots + J(x_N u_N) \\
       = \mid x_1 \mid J(u_1) + \cdots + \mid x_n \mid J(u_N) \\
       \leq \sqrt{\mid x_1 \mid^2 + \cdots + \mid x_N \mid^2} \sqrt{J{(u_1)}^2 + \cdots + J{(u_N)}^2}$ \\
       従って $M_2 = \sqrt{J{(u_1)}^2 + \cdots + J{(u_N)}^2}$とおくと \\
       $J(x)\leq M_2 \parallel x \parallel_2$が成り立つ

       (ii) 任意の$x, y \in X$に対して\\
       $ 0 \leq \mid J(x) - J(y) \mid \\ \leq J(x - y) \\ \leq M_2 \parallel x - y \parallel_2$ \\
       よって$x \rightarrow y \Rightarrow J(x) \rightarrow J(y)$であるので \\ $J: X \rightarrow \mathbf{R}$はノルム空間$(X, \parallel\bullet\parallel_2)$上の連続関数\\
       ここで、$\tilde{J}: \mathbf{R} \rightarrow \mathbf{R}$を\\
      $\tilde{J}(x_1, \ldots ,x_N) = J(x_1 u_1 + \cdots + x_N u_N)$と定義すると$\tilde{J}$は$(\mathbf{R}^N, d_2)$上の連続関数

      (iii) $S = \lbrace x \in X \mid \parallel x \parallel_2 =1 \rbrace \\ \tilde{S} = \lbrace (x_1, \ldots, x_N) \in \mathbf{R}^N \mid \sqrt{x_1^2 + \cdots + x_N^2} = 1 \rbrace$とすると \\
      $\tilde{S}$は$(\mathbf{R}^N, d_2)$における有界閉集合であるので、定理2.3 (b) (Heine-Borelの被覆定理)よりコンパクト集合。\\
      よって定理2.5より、$\tilde{J}$は$\tilde{S}$上で最小値$M_1$をもつ、従って$J$は$S$上で最小値$M_1$を持つ。\\
      $x\in S$のとき、$x \neq 0$であるので$J(x) > 0$よって$M_1 > 0$

      以上から
      \begin{itemize}
        \item $x \neq 0$のとき \\
        $M_1 \leq J (\frac{1}{\parallel x \parallel_2} x) = \frac{1}{\parallel x \parallel_2} x$ \\
          従って$M_1\parallel x \parallel_2 \leq J(x)$
        \item $x = 0$のとき \\ $M_1 \parallel x\parallel_2 = 0 = J(x)$
      \end{itemize}
\end{document}
